\documentclass[12pt]{article}
\usepackage[paperheight=25cm,paperwidth=17cm,top=1.5cm,bottom=1.cm,left=1cm,right=1cm]{geometry}
\usepackage{fancyhdr}
\renewcommand{\subsectionmark}[1]{\markright{#1}}

\fancypagestyle{main}{
\fancyhf{}
\addtolength{\topmargin}{-3.18243pt}
\setlength{\headsep}{3pt}
\fancyhead[R]{  % right header mark
\begin{tikzpicture} 
\node[font=\bfseries] at (0,0) {\footnotesize\thepage}; 
\end{tikzpicture}
}  % right header mark
}

\renewcommand{\footrulewidth}{0pt}% default is 0pt
\renewcommand{\headrulewidth}{0pt}% default is 0pt

\renewcommand{\headrulewidth}{0pt}
\setlength{\headheight}{23.5pt}
%\usepackage{mathptmx}
\usepackage{domitian}
\usepackage[OT1]{fontenc}
\DeclareMathAlphabet{\mathcal}{OMS}{cmsy}{m}{n}

\usepackage[tbtags]{amsmath}
\usepackage{amssymb}
\allowdisplaybreaks
\usepackage{mathtools}
\usepackage{hyperref}
\usepackage{bm}
\usepackage{fdsymbol}
\usepackage[skip=3ex, indent=0pt]{parskip}
\usepackage{tensor}
\usepackage{physics}
%\usepackage{physics2}
%\usephysicsmodule{ab}
\usepackage{url}
\usepackage[shortlabels]{enumitem}
\usepackage{tikz}
\usetikzlibrary{matrix,fit}
\NewDocumentCommand{\mtrx}{m}{
\begin{bmatrix}#1\end{bmatrix}
}
\def\vx{\vb*{x}}
\def\vy{\vb*{y}}
\def\vz{\vb*{z}}
\def\cM{{\cal M}}
\def\cX{{\cal X}}
\def\cB{{\cal B}}
\def\cV{{\cal V}}
\def\df{{\sf d}}
\def\R{\mathbb{R}}
\def\N{\mathbb{N}}
\def\F{\mathbb{F}}

\begin{document}
\pagestyle{main}
Define a bilinear map $E: \cV \times \cV \to \cV$ with the following properties:
\begin{enumerate}[i.,itemsep=0ex,parsep=1ex]
    \item Bilinear property: $E(a\vx+b\vy,\vz)=aE(\vx,\vz)+bE(\vy,\vz)$,
    $E(\vx,a\vy+b\vz)=aE(\vx,\vy)+bE(\vx,\vz)$ where $\vx,\vy,\vz \in \cV$ and $a,b \in \F$. \label{num:bilinear}
    \item Antisymmetric property: $E(\vx,\vy)=-E(\vy,\vx)$. \label{num:antisymmetric}
    \item Orthogonal property: $g(E(\vx,\vy),\vx)=g(E(\vx,\vy),\vy)=0$ where $g: \cV \times \cV \to \F$ is the non-degenerate, bilinear form that defines the inner product between two vectors. \label{num:orthogonal}
\end{enumerate}
\end{document}