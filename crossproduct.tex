\documentclass[12pt]{article}
\usepackage{geometry}
%\usepackage[paperheight=25cm,paperwidth=17cm,top=1.5cm,bottom=1.cm,left=7cm,right=7cm]{geometry}
\usepackage{fancyhdr}
\renewcommand{\subsectionmark}[1]{\markright{#1}}

\usepackage{ifdraft}

\ifoptiondraft
{
    \usepackage{showkeys}
    \geometry{paperheight=25cm,paperwidth=29cm,top=1.5cm,bottom=1.cm,left=7cm,right=7cm}
}
{
    \geometry{paperheight=25cm,paperwidth=17cm,top=1.5cm,bottom=1.cm,left=1cm,right=1cm}
}

\fancypagestyle{main}{
\fancyhf{}
\addtolength{\topmargin}{-3.18243pt}
\setlength{\headsep}{3pt}
\fancyhead[R]{  % right header mark
\begin{tikzpicture} 
\node[font=\bfseries] at (0,0) {\footnotesize\thepage}; 
\end{tikzpicture}
}  % right header mark
}

\renewcommand{\footrulewidth}{0pt}% default is 0pt
\renewcommand{\headrulewidth}{0pt}% default is 0pt

\renewcommand{\headrulewidth}{0pt}
\setlength{\headheight}{23.5pt}
\usepackage{mlmodern}
\usepackage[T1]{fontenc}
%\usepackage{newtxtext,newtxmath}
\DeclareMathAlphabet{\mathcal}{OMS}{cmsy}{m}{n}

\usepackage[tbtags]{amsmath}
\usepackage{amssymb}
\allowdisplaybreaks
\usepackage{mathtools}

%This is for hypertext references
\usepackage{url}%this line and the next are related to hyperlinks
\usepackage[colorlinks=true, pdfstartview=FitV, linkcolor=blue, citecolor=blue, urlcolor=blue]{hyperref}

\usepackage{bm}
%\usepackage{fdsymbol}
\usepackage[skip=3ex, indent=0pt]{parskip}
\usepackage{tensor}
\usepackage{physics}
%\usepackage{physics2}
%\usephysicsmodule{ab}
\usepackage{url}
\usepackage[shortlabels]{enumitem}
\usepackage{tikz}
\usetikzlibrary{matrix,fit}
\NewDocumentCommand{\mtrx}{m}{
\begin{bmatrix}#1\end{bmatrix}
}
\def\ve{\vb*{e}}
\def\vh{\vb*{h}}
\def\vu{\vb*{u}}
\def\vx{\vb*{x}}
\def\vy{\vb*{y}}
\def\vz{\vb*{z}}
\def\cM{{\cal M}}
\def\cX{{\cal X}}
\def\cB{{\cal B}}
\def\cV{{\cal V}}
\def\df{{\sf d}}
\def\R{\mathbb{R}}
\def\N{\mathbb{N}}
\def\F{\mathbb{F}}
\DeclareMathOperator{\End}{end}

\begin{document}
\pagestyle{main}
\begin{center}
    \Large\bfseries Cross Products
\end{center}
Most of the material presented here is taken from \cite{elduque2004vector}.

Given a finite dimensional vector space $\cV$ over a field $\F$ with characteristic zero and with an inner product $g: \cV \times \cV \to \F$ 
define a bilinear map $E: \cV \times \cV \to \cV$ with the following properties:
\begin{enumerate}[label={\Roman*.},ref={\Roman*},itemsep=0ex,parsep=1ex]
    \item Bilinear property: $E(a\vx+b\vy,\vz)=aE(\vx,\vz)+bE(\vy,\vz)$,
    $E(\vx,a\vy+b\vz)=aE(\vx,\vy)+bE(\vx,\vz)$ where $\vx,\vy,\vz \in \cV$ and $a,b \in \F$. \label{num:bilinear}
    \item Antisymmetric property: $$g\qty(E(\vx,\vy),E(\vx,\vy))=\mqty|g(\vx,\vx) & g(\vx,\vy) \\ g(\vy,\vx) & g(\vy,\vy)|$$ for all $\vx,\vy \in \cV$. 
    This is an expression of the antisymmetric property since \label{num:antisymmetric}
    \begin{align*}
    g\qty(E(\vx,\vy),E(\vy,\vx)) & =\mqty|g(\vx,\vy) & g(\vx,\vx) \\ g(\vy,\vy) & g(\vy,\vx)| 
    =-\mqty|g(\vx,\vx) & g(\vx,\vy) \\ g(\vy,\vx) & g(\vy,\vy)| \\
    & =-g\qty(E(\vx,\vy),E(\vx,\vy))=g\qty(E(\vx,\vy),-E(\vx,\vy))
    \end{align*}
    and so we must have $E(\vx,\vy)=-E(\vy,\vx)$.
    \item Orthogonal property: $g(E(\vx,\vy),\vx)=g(E(\vx,\vy),\vy)=0$. \label{num:orthogonal}
\end{enumerate}
We can use the bilinear map $E$ to induce a linear map $L_x: \cV \to \cV$ as follows: 
\[
L_x(\vy)=E(\vx,\vy), \; \text{for all $\vy \in \cV$}.
\]
This follows from property \ref{num:bilinear}. We note that 
\[
L_x^2(\vy)=L_xE(\vx,\vy)=E(\vx,E(\vx,\vy)).
\]
From property \ref{num:antisymmetric} we have  
\begin{align*}
g(E(\vx,E(\vx,\vy)),E(\vx,E(\vx,\vy))) & = \mqty|g(\vx,\vx) & g(\vx,E(\vx,\vy)) \\ g(E(\vx,\vy), \vx) & g(E(\vx,\vy),E(\vx,\vy))| \\
& = \mqty|g(\vx,\vx) & 0 \\ 0 & g(E(\vx,\vy),E(\vx,\vy))| \\
& = g(\vx,\vx) g(E(\vx,\vy),E(\vx,\vy)) \\
& = g(\vx,\vx) \qty(g(\vx,\vx)g(\vy,\vy)-g(\vx,\vy)^2) \\
& = g(\vx,\vx)^2 g(\vy,\vy) - g(\vx,\vy)^2 g(\vx,\vx) \\
& = g(\vx,\vx)^2 g(\vy,\vy) + g(\vx,\vy)^2 g(\vx,\vx) \\
& \kern2ex - 2 g(\vx,\vy)^2 g(\vx,\vx) \\
& = g(g(\vx,\vy)\vx-g(\vx,\vx)\vy,g(\vx,\vy)\vx-g(\vx,\vx)\vy).
\end{align*}
Therefore 
\[
E(\vx,E(\vx,\vy)) = g(\vx,\vy)\vx-g(\vx,\vx)\vy
\]
and
\[
L_x^2(\vy)=g(\vx,\vy)\vx-g(\vx,\vx)\vy.
\]

The tensor product $\vx \otimes \vy$ induces a linear map $T_{x \otimes y}: \cV \to \cV$ defined as 
\begin{align*}
T_{x \otimes y}(\vz) & =  g(\vy,\vz) \vx, \; \text{for all $\vz \in \cV$}. 
\end{align*}
From these definitions we can also generate the composite maps 
\[
T_{x \otimes y}L_z (\vu) = T_{x \otimes y} (E(\vz,\vu))=g(\vy,E(\vz,\vu))\vx 
\]
and
\begin{equation}
L_{x}T_{y \otimes z} (\vu) = L_{x}(g(\vz,\vu)\vy)=g(\vz,\vu)E(\vx,\vy)=T_{E(x,y)\otimes z} \vu.
\label{eq:xcrossytensorz}
\end{equation}
Note that since 
\[
g(\vx+\vy,E(\vx+\vy,\vz)) = 0
\]
(using property \ref{num:orthogonal}) we have 
\begin{align*}
    g(\vx+\vy,E(\vx+\vy,\vz)) & = g(\vx+\vy,E(\vx,\vz)+E(\vy,\vz)) \\
    & = g(\vx,E(\vx,\vz))+g(\vy,E(\vx,\vz))+g(\vx,E(\vy,\vz))+g(\vy,E(\vy,\vz)) \\
    & =  g(\vy,E(\vx,\vz)) + g(\vx,E(\vy,\vz)) = 0 .
\end{align*}
So we obtain 
\[
g(\vy,E(\vz,\vx)) = g(\vx,E(\vy,\vz))
\]
using the antisymmetric property of $E$. In the same way we can prove that this property is cyclic, i.e.
\begin{equation} \label{eq:cyclic}
    g(\vx,E(\vy,\vz)) = g(\vz,E(\vx,\vy)) = g(\vy,E(\vz,\vx)) .
\end{equation}
Using this property $g(\vy,E(\vz,\vu)) = g(\vu,E(\vy,\vz))$ and so
\begin{equation}
T_{x \otimes y}L_z (\vu) = g(\vu,E(\vy,\vz))\vx = T_{x \otimes E(y,z)} (\vu).
\label{eq:xtensorycrossz}
\end{equation}

Note that 
\[
L_x^2(\vy)=g(\vx,\vy)\vx-g(\vx,\vx)\vy = T_{x \otimes x} (\vy) - g(\vx,\vx) I(\vy)
\]
where $I: \cV \to \cV$ is the identity map. So 
\begin{equation} \label{eq:Lx2}
L_x^2 = T_{x \otimes x} - g(\vx,\vx) I.
\end{equation} 
We can obtain the linearization of $L_x^2$ by writing 
\begin{align*}
    L_{x+h}^2(\vy) & = g(\vx+\vh,\vy)(\vx+\vh) - g(\vx+\vh,\vx+\vh) \vy \\
    L_{x+h}^2 & = L_{x+h}L_{x+h} = L_x^2 + L_x L_h + L_h L_x + L_h^2
\end{align*}
The r.h.s. of the first equation expands to 
\begin{align*}
    L_{x+h}^2(\vy) & = g(\vx,\vy)\vx - g(\vx,\vx) \vy + g(\vx,\vy) \vh - g(\vx,\vh) \vy \\
    & \kern2ex + g(\vh,\vy)\vx - g(\vh,\vx) \vy + g(\vh,\vy)\vh - g(\vh,\vh) \vy \\
    & = L_x^2(\vy) + g(\vx,\vy) \vh - g(\vx,\vh) \vy + L_h^2(\vy). 
\end{align*}
We obtain 
\[
L_x L_h (\vy) + L_h L_x (\vy) = g(\vx,\vy)\vh - g(\vx,\vh) \vy + g(\vh,\vy)\vx - g(\vh,\vx)\vy. 
\]
If 
\begin{equation} \label{eq:linearization}
L_x L_h (\vy) = g(\vx,\vy)\vh - g(\vx,\vh) \vy
\end{equation}
then 
\[
L_h L_x (\vy) = g(\vh,\vy)\vx - g(\vh,\vx)\vy.
\]
Note that this linearization agrees with the property 
\begin{align*}
L_x L_h (\vy) & = g(\vx,\vy)\vh - g(\vx,\vh) \vy \\
& = - \qty(g(\vx,\vh) \vy - g(\vx,\vy)\vh) \\
& = - L_x L_y (\vh).
\end{align*}

We can use the linearization of $L_x^2$ to complete the following derivation:
\begin{align*}
    \qty(L_{E(x,y)} + L_x L_y)(\vz) & =E(E(\vx,\vy),\vz) +L_x L_y (\vz)  \\
    & = -E(\vz,E(\vx,\vy))+ L_x L_y (\vz) \\
    & = - L_z L_x (\vy) +  L_x L_y (\vz) \\ 
    & = - \qty(g(\vz,\vy)\vx - g(\vz,\vx)\vy) + g(\vx,\vz)\vy - g(\vx,\vy) \vz \\
    & = 2 g(\vx,\vz) \vy - g(\vz,\vy)\vx - g(\vx,\vy) \vz.
\end{align*}
This result can be used to write 
\begin{equation}
    L_{E(x,y)} + L_x L_y = 2  T_{y \otimes x} - T_{x \otimes y} - g(\vx,\vy) I. \label{eq:Exy_xy}
\end{equation}
From this equation we get 
\begin{align}
    L_{E(x,y)} L_x + L_x L_y L_x & = 2 T_{y \otimes x} L_x - T_{x \otimes y} L_x - g(\vx,\vy) L_x \nonumber\\
    L_x L_y L_x & = - L_{E(x,y)} L_x + 2 T_{y \otimes x} L_x - T_{x \otimes y} L_x - g(\vx,\vy) L_x \nonumber\\
    L_x L_y L_x & = - L_{E(x,y)} L_x  - T_{x \otimes E(y,x)} - g(\vx,\vy) L_x \label{eq:first}
\end{align}
since using (\ref{eq:xtensorycrossz})
\[
T_{y \otimes x} L_x (\vz) = T_{y \otimes E(x,x)} = 0.
\]
($E(x,x)=0$ follows from property \ref{num:antisymmetric}).
From (\ref{eq:Exy_xy}) 
\begin{align*}
    L_{E(x,y)} L_x & = - L_{E(E(x,y),x)} + 2 T_{x \otimes E(x,y)} - T_{E(x,y)\otimes x} - g(E(\vx,\vy),\vy) I \\
    & = - L_{E(E(x,y),x)} + 2 T_{x \otimes E(x,y)} - T_{E(x,y)\otimes x} ;
\end{align*}
this produces a second equation for $L_xL_yL_x$:
\begin{align}
    L_x L_y L_x & = - \qty(- L_{E(E(x,y),x)} + 2 T_{x \otimes E(x,y)} - T_{E(x,y)\otimes x} ) -  T_{x \otimes E(y,x)} - g(\vx,\vy) L_x \nonumber\\
    & = L_{E(E(x,y),x)} - 2 T_{x \otimes E(x,y)} + T_{E(x,y)\otimes x} -  T_{x \otimes E(y,x)} - g(\vx,\vy) L_x \nonumber\\
    & = L_{E(E(x,y),x)} - 2 T_{x \otimes E(x,y)} + T_{E(x,y)\otimes x} +  T_{x \otimes E(x,y)} - g(\vx,\vy) L_x  \nonumber\\
    & = L_{E(E(x,y),x)} -  T_{x \otimes E(x,y)} + T_{E(x,y)\otimes x} - g(\vx,\vy) L_x. \label{eq:second}
\end{align}
Next 
\[
E(E(\vx,\vy),\vx)=-E(\vx,E(\vx,\vy))=-\qty(g(\vx,\vy)\vx-g(\vx,\vx)\vy)=g(\vx,\vx)\vy-g(\vx,\vy)\vx.
\]
Therefore
\begin{align*}
L_{E(E(x,y),x)}(\vz) & = E\qty(g(\vx,\vx)\vy-g(\vx,\vy)\vx,\vz)=g(\vx,\vx)E(\vy,\vz)-g(\vx,\vy)E(\vx,\vz) \\
& =  g(\vx,\vx) L_y(\vz) - g(\vx,\vy) L_x(\vz).
\end{align*}
The third equation for $L_xL_yL_x$ is:
\begin{align}
    L_x L_y L_x & = g(\vx,\vx) L_y -  g(\vx,\vy) L_x - T_{x \otimes E(x,y)} + T_{E(x,y)\otimes x} - g(\vx,\vy) L_x \nonumber\\
    & = g(\vx,\vx) L_y - 2  g(\vx,\vy) L_x - T_{x \otimes E(x,y)} + T_{E(x,y)\otimes x} . \label{eq:third}
\end{align}

We choose an orthonormal basis $\qty{\ve_i}_{i=1}^d$ where $\dim \cV = d$. Define the following linear map $S: \End(\cV) \to \End(\cV)$
\[
f \mapsto \sum_{i=1}^d L_{e_i} \circ f \circ L_{e_i}.
\]
If $f=I$ then using (\ref{eq:Lx2})
\[
S(I)=\sum_{i=1}^d L_{e_i}^2 = \sum_{i=1}^d T_{e_i \otimes e_i} - \sum_{i=1}^d I = \sum_{i=1}^d T_{e_i \otimes e_i} -d I.
\]
Since 
\[
\sum_{i=1}^d T_{e_i \otimes e_i} (\vx)= \sum_{i=1}^d g(\ve_i,\vx) \ve_i = \vx ,
\]
\begin{equation}
S(I) = (1-d)I.  \label{eq:SI}
\end{equation}

If $f=T_{x \otimes y}$ then 
\begin{align*}
    S(T_{x \otimes y}) & = \sum_{i=1}^d L_{e_i} \circ T_{x \otimes y}  \circ L_{e_i} \\
    & = \sum_{i=1}^d L_{e_i} \circ T_{x \otimes E(y,e_i)} & \text{using (\ref{eq:xtensorycrossz})}\\
    & = \sum_{i=1}^d T_{E(e_i,x)\otimes E(y,e_i)} & \text{using (\ref{eq:xcrossytensorz})}.
\end{align*}
Using (\ref{eq:cyclic})
\begin{align*}
T_{E(e_i,x)\otimes E(y,e_i)} (\vz) & = g(E(\vy,\ve_i),\vz) E(\ve_i,\vx) \\
& =  g(\vz,E(\vy,\ve_i)) E(\ve_i,\vx) \\
& = g(\ve_i,E(\vz,\vy)) E(\ve_i,\vx) \\
& = g(\ve_i,E(\vy,\vz)) E(\vx,\ve_i) \\
& = E\qty(\vx,g(\ve_i,E(\vy,\vz)) \ve_i)
\end{align*}
which substituted back in the summation gives
\begin{align*}
    S(T_{x \otimes y}) & = \sum_{i=1}^d E\qty(\vx,g(\ve_i,E(\vy,\vz)) \ve_i) \\
    & = E\qty(\vx,\sum_{i=1}^d g(\ve_i,E(\vy,\vz)) \ve_i) \\
    & = E(\vx,E(\vy,\vz)) = L_x L_y (\vz)
\end{align*}
and so
\begin{equation}
    S(T_{x \otimes y}) = L_x L_y \label{eq:Sxoy}.
\end{equation}

If $f=L_{y}$ then using (\ref{eq:third})
\begin{align*}
    S(L_y) & = \sum_{i=1}^d L_{e_i} \circ L_{y} \circ L_{e_i} \\
    & = \sum_{i=1}^d \qty(g(\ve_i,\ve_i) L_y - 2 g(\ve_i,\vy) L_{e_i} - T_{e_i \otimes E(e_i,y)} + T_{E(e_i,y) \otimes e_i}).
\end{align*}
Since $g(\ve_i,\ve_i)=1$ the first term is simply $d L_y$; for the second term write 
\begin{align*}
\sum_{i=1}^d g(\ve_i,\vy) L_{e_i} (\vz) & = \sum_{i=1}^d g(\ve_i,\vy) E(\ve_i,\vz) = \sum_{i=1}^d E\qty(g(\ve_i,\vy)\ve_i,\vz) \\
& = E\qty(\sum_{i=1}^d g(\ve_i,\vy)\ve_i,\vz) = E(\vy,\vz) = L_y(\vz).
\end{align*}
For the third term 
\begin{align*}
\sum_{i=1}^d T_{e_i \otimes E(e_i,y)} (\vz) & = \sum_{i=1}^d g(\vz,E(\ve_i,\vy)) \ve_i = \sum_{i=1}^d g(\ve_i,E(\vy,\vz)) \ve_i \\
& = E(\vy,\vz) = L_y (\vz).
\end{align*}
In a similar way the fourth term is 
\begin{align*}
    \sum_{i=1}^d  T_{E(e_i,y) \otimes e_i} (\vz) & = \sum_{i=1}^d  E(\ve_i,\vy) g(\vz,\ve_i) \\
    & = E\qty(\sum_{i=1}^d g(\vz,\ve_i) \ve_i, \vy) = E(\vz,\vy) = - E(\vy,\vz) = - L_y (\vz).
\end{align*}
Hence we conclude that 
\begin{equation}
    S(L_y) = (d - 2 - 1 -1) L_y=(d-4)L_y  \label{eq:Sy}.
\end{equation}

If $f=L_x L_y$ then 
\begin{align}
    S(L_xL_y) & = - S(L_{E(x,y)}) + 2 S(T_{y \otimes x}) - S(T_{x \otimes y}) - g(\vx,\vy) S(I)  && \text{using (\ref{eq:Exy_xy})} \nonumber \\
    & = -(d-4)L_{E(x,y)} + 2 S(T_{y \otimes x}) - S(T_{x \otimes y}) - g(\vx,\vy) S(I)  && \text{using (\ref{eq:Sy})} \nonumber \\
    & = -(d-4)L_{E(x,y)} + 2 L_y L_x - L_x L_y - g(\vx,\vy) S(I)  && \text{using (\ref{eq:Sxoy})} \nonumber \\
    & = -(d-4)L_{E(x,y)} + 2 L_y L_x - L_x L_y - (1-d) g(\vx,\vy) I  && \text{using (\ref{eq:SI})} . \label{eq:Sxy} 
\end{align}

Using these derivations we can compute the following transformation: 
\[
g = \sum_{i=1}^d \sum_{j=1}^d L_{e_i} \circ L_x \circ L_{e_j} \circ L_{e_i} \circ L_{e_j}. 
\]
First using $S(L_{e_i})=\sum_{j=1}^d L_{e_j} \circ L_{e_i} \circ L_{e_j}$ and (\ref{eq:Sy}) we have 
\[
g = \sum_{i=1}^d  L_{e_i} \circ L_x \circ (d-4) L_{e_i} = (d-4) \sum_{i=1}^d  L_{e_i} \circ L_x \circ L_{e_i} = (d-4) S(L_x) = (d-4)^2 L_x.
\]
Another way of computing the same transformation is to write 
\begin{align*}
g & = \sum_{j=1}^d S(L_xL_{e_j}) \circ L_{e_j}  \\
& = \sum_{j=1}^d \qty(-(d-4)L_{E(x,\ve_j)} + 2 L_{e_j} L_x - L_x L_{e_j} - (1-d) g(\vx,\ve_j) I) \circ L_{e_j}.
\end{align*}
using (\ref{eq:Sxy}).
Starting from the last term we have 
\[
g(\vx,\ve_j) I \circ L_{e_j} (\vz) = g(\vx,\ve_j) L_{e_j} (\vz) = g(\vx,\ve_j) E(\ve_j,\vz)
\]
which after applying the summation w.r.t. $j$ becomes 
\[
\sum_{j=1}^d g(\vx,\ve_j) I \circ L_{e_j} (\vz) = E\qty(\sum_{j=1}^d g(\vx,\ve_j) \ve_j, \vz ) = E(\vx,\vz) = L_x(\vz).
\]
For $L_{e_j} L_x $ using (\ref{eq:Sy}) we have 
\[
2 \sum_{j=1}^d  L_{e_j} \circ L_x \circ L_{e_j} = 2 S(L_x) = 2 (d-4) L_x 
\]
while for $L_x L_{e_j}$ we have using (\ref{eq:SI})
\[
L_x \sum_{j=1}^d L_{e_j} \circ L_{e_j} = L_x \sum_{j=1}^d L_{e_j} \circ I \circ L_{e_j} = L_x S(I) = (1-d)L_x.
\]
For $L_{E(x,\ve_j)}$ write using (\ref{eq:first})
\begin{align*}
L_{E(x,e_j)} L_{e_j}  & = - L_{E(e_j,x)} L_{e_j} = - \qty(-L_{e_j}L_x L_{e_j} - L_{e_j \otimes E(x,e_j)} - g(\ve_j,\vx) L_{e_j}) 
\end{align*}
For the second term 
\[
L_{e_j \otimes E(x,e_j)} (\vz) = g(\vz,E(\vx,\ve_j)) \ve_j = g(\ve_j,E(\vz,\vx)) \ve_j
\]
which after summation w.r.t. $j$ is equal to $E(\vz,\vx)$. For the third term
\[
g(\ve_j,\vx) L_{e_j} (\vz) = g(\ve_j,\vx) E(\ve_j,\vz) = E\qty(g(\ve_j,\vx)\ve_j,\vz)
\]
which after summation w.r.t. $j$ is equal to $E(\vx,\vz)$. Therefore 
\begin{align*}
\sum_{j=1}^d L_{E(x,e_j)} L_{e_j}  & =  \sum_{j=1}^d L_{e_j}L_x L_{e_j} +  E(\vz,\vx) + E(\vx,\vz) = S(L_x) = (d-4)L_x 
\end{align*}
and so computing the transformation using this method we have 
\begin{align*}
g & = \sum_{j=1}^d \qty(-(d-4)L_{E(x,\ve_j)} + 2 L_{e_j} L_x - L_x L_{e_j} - (1-d) g(\vx,\ve_j) I) \circ L_{e_j}  \\
& = -(d-4) \underbrace{\sum_{j=1}^d L_{E(x,\ve_j)} L_{e_j}}_{(d-4)L_x} + 2  \underbrace{\sum_{j=1}^d L_{e_j} L_x L_{e_j}}_{ (d-4) L_x}
- \underbrace{L_x \sum_{j=1}^d L_{e_j} L_{e_j}}_{(1-d)L_x} -(1-d) \underbrace{\sum_{j=1}^d g(\vx,\ve_j) I L_{e_j}}_{L_x} \\
& = \qty(-(d-4)^2+2(d-4)-(1-d)-(1-d))L_x \\
& = \qty(-(d-4)^2+2(d-4)-2(1-d))L_x
\end{align*}
Since both methods compute the same transformation we must have 
\[
-(d-4)^2+2(d-4)-2(1-d) = (d-4)^2
\]
or 
\[
(d-4)^2 - (d-4) - (d-1) = d^2 - 10 d + 21 = (d-3)(d-7)=0.
\]

For a vector space $\cV$ with $\dim \cV = 3$, given a linear map $\phi: \cV \to \cV$, define a linear map $\bar \phi \in \End(\cV)$
\[
\bar \phi E(\vx,\vy) = E(\phi\vx,\phi\vy).
\]
In this vector space the \emph{volume form} is a (0,3) antisymmetric tensor
\[
\omega(\vx_1,\vx_2,\vx_3) = \epsilon_\sigma \omega(\vx_{\sigma(1)},\vx_{\sigma(2)},\vx_{\sigma(3)}) 
\]
where $\sigma \in S_3$ the symmetry group of permutations of $\{1,2,3\}$ and $\epsilon_\sigma=(-1)^{N(\sigma)}$
where $N(\sigma)$ is the number of single transpositions that generate $\sigma$.  
Define the volume form: 
\begin{equation} \label{eq:R3volumeform}
\omega(\vx_1,\vx_2,\vx_3)= g(\vx_1,E(\vx_2,\vx_3));
\end{equation}
We can use (\ref{eq:cyclic}) to show that the definition is antisymmetric. For the symmetry group $S_3$, a cyclic transposition is 
even and this agrees with (\ref{eq:cyclic}). We are left with $f_1=\smqty(1 & 2 & 3 \\ 1 & 3 & 2)$, $f_2=\smqty(1 & 2 & 3 \\ 3 & 2 & 1)$
and $f_3=\smqty(1 & 2 & 3 \\ 2 & 1 & 3)$. These are all odd permutations and, 
\begin{align*}
\omega(\vx_1,\vx_3,\vx_2) & =g(\vx_1,E(\vx_3,\vx_2))=-g(\vx_1,E(\vx_2,\vx_3)) \\
& =-\omega(\vx_1,\vx_2,\vx_3), \\
\omega(\vx_3,\vx_2,\vx_1) & =g(\vx_3,E(\vx_2,\vx_1))=-g(\vx_3,E(\vx_1,\vx_2)) \\
& =-g(\vx_1,E(\vx_2,\vx_3)) =-\omega(\vx_1,\vx_2,\vx_3), \\
\omega(\vx_2,\vx_1,\vx_3) & =g(\vx_2,E(\vx_1,\vx_3))=-g(\vx_2,E(\vx_3,\vx_1)) \\
& =-g(\vx_1,E(\vx_2,\vx_3)) =-\omega(\vx_1,\vx_2,\vx_3).
\end{align*}
Given a volume form the definition of the determinant is
\[
\det(\phi) = \frac{\omega(\phi \ve_1, \phi \ve_2, \phi \ve_2)}{\omega(\ve_1,\ve_2,\ve_3)};
\]
where $\{ \ve_i \}_{i=1}^3$ is a basis of $\cV$.
This is unique since the vector space of volume forms has dimension 1. This definition conforms with all 
the properties of a determinant. If $\phi$ is singular then $\{ \phi \ve_1, \phi \ve_2, \phi \ve_2\}$ are not
independent vectors and so $\det(\phi)=0$ since $\omega(\phi \ve_1, \phi \ve_2, \phi \ve_2)=0$. 
For two non-singular endomorphisms $\phi,\psi$
\begin{align*}
\det(\phi\psi) & = \frac{\omega(\phi \psi\ve_1, \phi \psi\ve_2, \phi \ve_2)}{\omega(\ve_1,\ve_2,\ve_3)} \\
& = \frac{\omega(\phi \psi\ve_1, \phi \psi\ve_2, \phi \ve_2)}{\omega(\psi\ve_1,\psi\ve_2,\psi\ve_3)}
\frac{\omega(\psi\ve_1,\psi\ve_2,\psi\ve_3)}{\omega(\ve_1,\ve_2,\ve_3)} = \det(\phi)\det(\psi)
\end{align*}
since $\{ \psi\ve_i \}_{i=1}^3$ is a basis of $\cV$.
Using the volume form definition from (\ref{eq:R3volumeform}) 
\begin{align*}
\det(\phi) & = \frac{g(\phi \ve_1, E(\phi \ve_2, \phi \ve_3) )}{g( \ve_1, E(\ve_2,\ve_3))} .
\end{align*}
Wlog, to simplify the derivations assume that $\{ \ve_i \}_{i=1}^3$ is the standard orthonormal basis 
with $E(\ve_1,\ve_2)=\ve_3$, $E(\ve_2,\ve_3)=\ve_1$ and $E(\ve_3,\ve_1)=\ve_2$.  So we can write
$$g( \ve_1, E(\ve_2,\ve_3))=g(\ve_1,\ve_1)=1.$$
From the definition of $\bar \phi$
\[
\det(\phi) = g(\phi \ve_1, \bar \phi E(\ve_2, \ve_3) ) = g(\phi \ve_1, \bar \phi \ve_1 ).
\]
The adjoint of $\phi$, $\phi^\ast$, is defined as the endomorphism with the property
\[
    g(\phi \ve_1, \bar \phi \ve_1 ) = g(\ve_1, \phi^\ast \bar \phi \ve_1 ) .
\] 
So we must have 
\[
    \det(\phi) =   g(\ve_1, \phi^\ast \bar \phi \ve_1 ) 
\]
which holds only if 
\[
    \phi^\ast \bar \phi = \det(\phi) I.
\]
This can be rewritten as 
\[
    \bar \phi = \det(\phi) \qty(\phi^\ast)^{-1}.
\]
For a vector space like $\R^3$ we note that $\phi^\ast \equiv \phi^{\sf T}$ and so this equation 
can be rewritten as $\bar \phi = \det(\phi) \phi^{-{\sf T}}$.

\bibliographystyle{plain} % We choose the "plain" reference style
\bibliography{crossproductreferences} 
\end{document}